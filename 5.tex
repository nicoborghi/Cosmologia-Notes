\chapterimage{/5/head.jpg} % Chapter heading image
\chapter{Storia Cronologica dell'Universo}\label{5:ch}

Non ci si avventurerà nella selva oscura della teoria GUT, ma verranno soltanto citati graficamente i fenomini-chiave e i principali periodi che caratterizzano le primissime fasi di vita del nostro universo. A partire dal tempo di Plank, $t_P=10^{-43}$ s, gli effetti quantistici diminuiscono col passare degli istanti, ma leptoni e adroni sono ancora la stessa cosa. Per questo motivo in questa fase può essere violata la conservazione del numero barionico, ma essendo in equilibrio si ha: $n_b = n_{\overbar{b}}$ e qualunque eventuale eccesso può essere cancellato da reazioni che violano il numero barionico. La transizione GUT è l'unico momento in cui può essersi formata l'anisotropia di 1 parte su $10^8$ barioni (i modelli di bariogenesi faticano ancora a trovare un valore così piccolo). Inoltre in questi istanti si formano le cariche, tra cui i monopoli magnetici. Durante la fase intermedia successiva, che dura $10^{26}$ secondi, la temperatura cala di $12$ dex e alla fine si ha $R_H = 1$ cm. La fisica delle particelle fatica a trovare candidati per questo range di temperature, per questo il periodo viene definito \textit{deserto di particelle}. La seconda transizione importante è quella elettrodebole, durante la quale i leptoni e il neutrino possono prendere massa. Infine all'ultima transizione quark-adroni si ha $R_H = 1$ km. Da questo punto si introducono i modelli inflazionari.
\begin{figure}[H]
    \centering
    \includegraphics[width=.55 \textwidth]{Pictures/5/fasiprimordiali.png}
    \label{fig:4}
\end{figure}


\section{Modello di Guth (\textit{old inflation})}
I primo modelli di inflazione (anni '80) si basavano sulla teoria delle transizioni di fase del primo tipo. La temperatura critica precedentemente definita, è in questo caso la temperatura della transizione GUT e $\phi$ viene chiamato \textbf{inflatone}. A $T=T_{GUT}$ si iniziano a sviluppare due nuove minimi allo stesso livello del vecchio e si viene quindi a creare una posizione di falso equilibrio o \textit{falso vuoto}. Al calare della temperatura il sistema non raggiunge istantaneamente la posizione di equilibrio a causa della barriera di potenziale dovuta alla forma di $\phi$. Quando la barriera di potenziale viene vinta il sistema è sovraraffreddato e, raggiungendo il nuovo equilibrio, il sistema libera il calore latente (\textbf{reheating}). È quest'ultimo processo che può spiegare la generazione e termalizzazione di un grande numero di particelle.

La densità di lagrangiana dell'inflatone è: $\mathcal{L}_\phi = 1/2 \dot{\phi}^2-V(\phi, T)$ dove $V(\phi, T)$ è il potenziale che corrisponde al campo scalare dell'inflatone e fa il gioco dell'energia libera. Il contributo dell'inflatone al tensore energia-impulso vale:
\begin{equation*}
    T \phi_{ij}=-p_\phi g_{ij}+(p_\phi -\rho_\phi)~u_i u_j \qquad \left\{
        \def\arraystretch{1.5}
            \begin{array}{ll}
                p_\phi = \mathcal{L}_\phi = \frac{1}{2}\dot{\phi}^2-V \\
                \rho_\phi = \frac{1}{2}\dot{\phi}^2+V 
        \end{array}\right.
\end{equation*}

Tramite l'equazione di Eulero-Lagrange si ottiene l'\textbf{equazione fondamentale della dinamica dell'inflatone}:
\begin{equation}
    \ddot{\phi} + 3H \dot{\phi} + \frac{\partial V}{\partial \phi} =0
\end{equation}
in cui $3H\dot{\phi}$ gioca il ruolo di termine di frizione. Il "moto" dovrà essere rallentato affinché l'inflazione duri a sufficienza. Includendo questi contributi nelle equazioni di Friedmann (ponendo $c=1$), si può verificare che l'espansione risultante è accelerata:
\begin{equation*}
\left( \frac{\dot{a}}{a}\right)^2 = \frac{8\pi}{3}G\rho_\phi \qquad \rho_\phi \simeq V \qquad \rightarrow \qquad a=a_0 ~e^{t/\tau} \qquad \tau^{-1} = \sqrt{8\pi G V /3}
\end{equation*}
Inoltre, assumendo un valore del potenziale $V$ si può calcolare il numero di e-folding:
\begin{equation*}
N_{ef} = -8\pi G \int_{\phi_i}^{\phi_f} \left(\frac{\mathrm{d} \ln V}{\mathrm{d}\phi}\right)^{-1}\mathrm{d}\phi
\end{equation*}
che cresce, ovviamente, quanto più è piatto il potenziale. 

Il problema di questi modelli è che prevedono una dimensione di $R_H$ insufficiente per coprire l'intero universo, si creano delle "bolle di universo" troppo piccole. Si è quindi passati a considerare le transizioni del secondo tipo, nonostante non offrano le possibilità di un rilascio di energia sotto forma di calore latente. Queste transizioni prevedono un inizio istantaneo dell'inflazione, quindi la forma del potenziale dovrà essere regolata in modo tale che l'inflazione duri a sufficienza. Tali \textbf{new inflation models} non sono comunque apprezzati perché richiedono un \textit{fine-tuning} dei parametri.

\section{Modello di Inflazione Caotica}
Il paradigma attuale dell'inflazione, l'inflazione caotica, è dovuto a Linde e non si basa sulle transizioni di fase. È sufficiente che esista un campo scalare estremamente energetico e statico nella fase iniziale, $\phi$. L'equazione della dinamica, la densità e la pressione dell'inflatone sono le stesse.
\subsection{Potenziale quadratico}
Per semplificare i conti si utilizzano le unità planckiane (\ref{eq:unitaplanckiane}) e si assume un potenziale della forma:
$$
V(\phi)= \frac{m^2~\phi^2}{2}
$$
dove $m$ è la massa dello scalare corrispondente all'inflatone. L'equazione della dinamica dell'inflatone diventa:
\begin{equation}
    \ddot{\phi} + \sqrt{12 \pi} \left( \dot{\phi}^2+m^2 \phi^2 \right) \dot{\phi} + \frac{\partial V}{\partial \phi} =0
\end{equation}
Questa può essere studiata nello spazio delle fasi ($\phi, \dot{\phi}$) ed essendo del II ordine non autonoma vale la relazione $\ddot{\phi} = \dot{\phi} ~ \mathrm{d} \dot{\phi} /\mathrm{d} \phi$. Partendo con la condizione in cui il termine cinetico domina su quello potenziale, la traiettoria può essere studiata in tra fasi principali: caduta sull'attrattore, inflazione e graceful exit.

\vspace{1em}
\begin{example}[Caduta sull'attrattore] $\dot{\phi}^2 \gg m^2 \phi^2 \qquad p_\phi = \rho_\phi = \frac{\dot{\phi}}{2} \rightarrow w=1$ (\textit{ultra hard equation of state})
\end{example}
\noindent A differenza dei modelli visti in precedenza non è richiesto un particolare \textit{fine-tuning} dei parametri, ma si parte da generici valori iniziali $\dot{\phi} \gg \phi$. Dall'equazione della dinamica modificata si può ottenere l'andamento delle traiettorie in questo primo regime:
\begin{equation}
    -\dot{\phi} \frac{\mathrm{d} \dot{\phi}}{\mathrm{d} \phi} = \sqrt{12 \pi} \dot{\phi}^2 \quad \rightarrow \quad \dot{\phi} \propto e^{\sqrt{12 \pi} \phi}
\end{equation}
Integrandole in funzione del parametro tempo $t$:
\begin{equation}
    \phi= cost - \frac{\ln t}{\sqrt{12 \pi}} \qquad \dot{\phi} = \frac{1}{\sqrt{12 \pi}}\frac{1}{t};
\end{equation}
da cui è chiaro che in questa fase la traiettoria è praticamente parallela all'asse $\dot{\phi}$ ($\delta \phi \ll \delta\dot{\phi}$ ). Dalle equazioni di Friedmann si può inoltre verificare che l'andamento dei principali parametri cosmogici è lo stesso di un universo EdS con $w=1$:
\begin{equation}
    H = \frac{1}{3t} \qquad a \propto t^{1/3} \qquad \rho \propto a^{-6}
\end{equation}

In conclusione, se il campo scalare parte con un'energetica sufficiente, il sistema raggiunge naturalmente un $\phi$ poco differente dal valore iniziale a seguito di una forte diminuzione di $\dot{\phi}$.

\vspace{1em}
\begin{example}[Inflazione] 
    $\mathrm{d} \dot{\phi} / \mathrm{d} \phi=0 \qquad m\phi_i \gg \dot{\phi}_i$
\end{example}

Queste due assunzioni sono finalizzate al risultato finale, ossia quello di avere una fase inflazionaria sufficientemente lenta, in particolare è richiesto che il valore di inserzione $\phi_i$ sul cosiddetto \textbf{attrattore} sia sufficientemente alto. 
\begin{equation}
    \dot{\phi} =  -\frac{m}{\sqrt{12 \pi}} = cost \qquad \phi= \phi_i - \frac{m}{\sqrt{12 \pi}} (t-t_i) = \phi_f - \frac{m}{\sqrt{12 \pi}} (t-t_f); \label{eq:phidotphiinfl}
\end{equation}

Si assume $\phi_f = 0$, per cui il valore del potenziale sull'attrattore è $V= m^4 (t_f-t)^2 / 24 \pi$. L'inflazione termina quando $\ddot{a}=0$, ossia $\rho_\phi = -3p_\phi = m^2 / 8 \pi$. Dato che in questo regime $\rho_\phi \approx V$ si ha:
$$
\phi_f = \sqrt{\frac{1}{4\pi}}= \mathcal{O}(1)
$$
Il potenziale alla fine di questa fase è quindi dell'ordine di 1 volta il potenziale al tempo di Planck. Dalle equazioni di Friedmann si può inoltre calcolare l'andamento dei principali parametri cosmogici:
\begin{equation}
    H^2 = \frac{8\pi }{3}\rho_\phi \rightarrow H = \frac{m^2}{3} (t_f -t) \qquad a=a_f e^{-m^2 (t_f-t)^2 /6} \qquad a=a_i e^{(H+H_i)(t-t_i)^2 /2}
\end{equation}
Si nota che $H$ descresce in modo lineare nel tempo, mentre $a$ si espande in modo esponenziale: \textbf{inflazione}. Per risolvere i problemi dell'età dell'universo e dell'orizzonte si applica la condizione $\ln (a_f/a_i) \gg 60$ assumendo $\phi_f = 0$ e utilizzando l'equazione (\ref{eq:phidotphiinfl}):
\begin{equation}
    2 \pi \phi_i^2 \gg 60 \quad \rightarrow \quad \phi_i \gg 3 \div 4
\end{equation}
Questo significa avere valori di $\phi$ più grandi di quelli al tempo di Planck, ma la fisica è dettata da $V$. Per questo motivo, per evitare di entrare in regime quantistico, è necessario porre $V<1$. Questo si traduce in una condizione su $m$: $m^2 \phi^2 /2 <1$. Collocandosi a $m=m_{GUT}$ si deve avere $\phi_i < 10^4$ unità plankiane e per tale valore  $N_{ef} \gg 60$ è pienamente verificato.

\vspace{1em}
\begin{example}[Graceful Exit] 
    $H^2=\frac{4\pi }{ 3} \left(\dot{\phi} +m^2\phi^2\right) \qquad \ddot{\phi} + 3H \dot{\phi} + \frac{\partial V}{\partial \phi} =0$
\end{example}
In questa fase si utilizzano tutti i termini dell'equazione della dinamica. Dalla prima equazione: 
\begin{equation}
    \dot{\phi} \equiv \sqrt{\frac{3}{4\pi}} ~H \sin \theta, \quad m\phi \equiv \sqrt{\frac{3}{4\pi }} ~H \cos \theta \quad \rightarrow \quad 
\end{equation}
dove $\theta$ rappresenta l'angolo che descrive la traiettoria nello spazio ($\dot{\phi}, m\phi$). Derivando temporalmente le quantità $(H\sin\theta)$ e $(H\cos\theta)$ si ottengono le relazioni:
\begin{equation}
    \dot{H} = -3H^2 \sin^2 \theta \qquad\qquad  \dot{\theta} \propto -m \qquad \theta \propto -mt
\end{equation}
La derivata del parametro di Hubble oscilla nel tempo, ma è smorzata all'aumentare di H e l'angolo di fase varia linearmente. Integrando nel tempo si può ottenere il valore di $H$ e $a$ per $mt$ piccoli:
\begin{equation}
    H = \frac{2}{3t}\left( 1+ \mathrm{sinc} (2mt)\right) + \mathcal{O}(t^{-3}) \qquad\qquad a\propto t^{2/3}
\end{equation}
La $\mathrm{sinc}$ smorza l'ampiezza della spirale di H. Questo risultato non ci soddisfa particolarmente, perché usciremmo dall'inflazione con un'equazione di stato che non corrisponde alla componenete radiativa ($a_{EdS}\propto t^{1/2}$). Lo smorzamento sarà comunque il meccanismo che genererà le fluttuazioni di inflatone (particelle).

\begin{figure}[H]
    \centering
    \includegraphics[width=.8 \textwidth]{Pictures/5/chaosinfl.png}
    \caption{Diagramma di fase per il modello di inflazione caotica. Si distinguono le tre fasi principali: (1) caduta sull'attrattore, (2) inflazione, (3) graceful exit.}
\end{figure}



\subsection{Potenziali generalizzati}
Per forme del potenziale diverse da quella quadratica (bocciata dal satellite Plank) valgono comunque le equazioni:
\begin{equation*}
    \left\{\begin{matrix}
        \ddot{\phi} + 3H \dot{\phi} + \frac{\partial V}{\partial \phi} =0\\ 
        H^2 = 8\pi /3 \left(\dot{\phi}^2+V\right)
       \end{matrix}\right.
\end{equation*}
Per risolvere i problemi dell'orizzonte e della piattezza si introducono le condizioni di \textbf{slow rolling}:
\begin{equation}
    \left| \dot{\phi}^2 \right| \ll \left| V \right|  \qquad\qquad \left| \ddot{\phi} \right| \ll \left| 3H\dot{\phi} \right|;
\end{equation}
che corrispondono rispettivamente alle richieste che il termine potenziale sia dominante rispetto a quello cinetico e l'accelerazione domini sul termine di frizione. In questo modo si ottiene $a'/a = -8\pi V/V'$ (l'apice indica la derivata rispetto a $\phi$), da cui:
\begin{equation}
    60 \ll N_{ef} = 8\pi \int_\phi^{\phi_i} \frac{V}{V'} \mathrm{d}\phi. 
\end{equation}
In letteratura le condizioni di slow rolling vengono riscritte sotto forma di $V'$ e $V''$:
\begin{equation}
    \left| \left( V'/V\right)^2 \right| \ll 1 \qquad\qquad \left|  V''/V \right| \ll 1;
\end{equation}
e vengono introdotti i seguenti parametri:
\begin{equation}
    \varepsilon = \frac{1}{16\pi}\left(\frac{V'}{V}\right)^2 \qquad\qquad \eta = \frac{1}{8\pi}\left(\frac{V''}{V}\right)^2
\end{equation}
In generale bisogna avere $\varepsilon, \eta \ll 1$ (in unità planckiane), ma i valori precisi vengono calcolati per ogni nuovo modello, perché lasciano caratteristiche osservabili sulla CMB e nella struttura a larga scala (deviazioni dalla perfetta gaussianità). In pratica l'inflazione è falsificabile attraverso questi due numeretti. Un potenziale a legge di potenza $V=\lambda \phi^n / n$ ha le seguenti caratteristiche:
\begin{equation}
    \varepsilon =   \frac{1}{16\pi} \left( \frac{n}{\phi}\right)^2 \qquad \eta= 2 \frac{n-1}{n} ~\varepsilon \qquad\qquad a(\phi)=a_i ~e^{N_{ef}} = a_i ~e^{4\pi(\phi_i^2-\phi)/n}
\end{equation}
ossia restituisce sempre un'espansione esponenziale che può essere regolata tramite $n$ per ottenere l'$N_{ef}$ desiderato.

\subsection{Reheating}
La termalizzazione è possibile grazie all'elevata quantità di energia disponibile e può essere legata al decadimento dell'inflatone. Questo processo può avvenire in due modi:
$$
\phi \rightarrow \chi + \chi \qquad \phi \rightarrow \psi + {\overbar{\psi}}
$$
nel primo caso viene prodotta una coppia di scalari e nel secondo una coppia fermione-antifermione. La lagrangiana di interazione è $\Delta \mathcal{L} = -g \phi \chi^2 - h \phi \psi {\overbar{\psi}}$ dove $g$ e $h$ sono le costanti di accoppiamento dei due processi. I rispettivi tassi di decadimento sono dati dalle seguenti relazioni:
$$
\Gamma_\chi = g^2 / 8\pi m \qquad \Gamma =h^2 m / 8\pi 
$$
Per evitare di entrare nel regime quantistico bisogna che $g \lesssim m$ e $h\lesssim m^{1/2}$, quindi si possono assumere $\Gamma_\chi \approx m$ e $\Gamma_\psi \approx m^2$. Per cui a $m_{GUT}=10^{-4}$ si ha $\Gamma_\psi \ll \Gamma_\chi$ e domina il processo di decadimento degli scalari. La variazione di densità numerica degli scalari seguirà quindi la legge: $\dot{n}_\phi = - g^2 ~n_\phi / 8\pi m$. Il numero di oscillazioni dello scalare inflatone nell'unità di tempo ("giri di spirale") è: $N_{osc}=mt/2\pi$, per cui passando per il differenziale $\mathrm{d}n_\phi$ si ha:
\begin{equation}
    n_\phi \propto e^{-g^2 ~N_{osc}/4m^2}
\end{equation}
Sostituendo $g \sim m$ sono sufficienti pochissime oscillazioni per far decadere tutti gli inflatoni e formare gli scalari $\chi$. Anche per $g \ll m$ questo è verificato, grazie al contributo di processi di risonanza e condensazioni di Bose. Inoltre, assumendo che la massa dell'inflatone sia $m=10^{13}$ GeV ($10^{-6}$ unità planckiane):
\begin{equation}
    n_\phi (t_f)\approx \frac{1}{2}m\phi_f^2 \approx 10^{92}\; \mathrm{cm^{-3}}
\end{equation}
si ottengono un sacco di particelle e tutta l'entropia che potrebbe servire. Oggi l'energia dell'universo è talmente bassa che processi come questo non possono avere luogo, l'inflazione va collocata a livelli energetici alti.

\subsubsection{Temperatura di Reheating}
La temperatura in uscita del processo di inflazione deve necessariamente essere $T_{RH}<T_{GUT}\simeq 10^{16}$ GeV, altrimenti si riattraverserebbe ciclicamente la transizione GUT (pb. monopoli magnetici e cose di questo tipo). In ogni caso è necessario termalizzare l'universo, ossia $\Gamma_\chi^{-1}<H^{-1}$, nel limite (corrispondente alla fine del processo) si ha:
\begin{equation}
    \Gamma_\chi \simeq \frac{m}{8\pi} \quad\equiv\quad H \simeq \sqrt{\frac{8\pi^3}{90}g^* T^4}\quad \rightarrow \quad T_{RH} = 2\cdot 10^{-4} \left( g^*_{100} \right)^{-1/2} \left( m_{\phi, -6} \right)^{1/2}
\end{equation}
La cosiddetta temperatura di reheating dipende quindi dalla massa dell'inflatone, nel caso in cui $m_\phi = 10^{-6}$ si avrebbe $T_{RH}\approx 10^{-4}\;\mathrm{T_P}=10^{15}$ GeV che è già pericoloso! 

\newpage
\subsubsection{Sommario}
Il paradigma dell'inflazione si aggiunge al modello del Big Bang per risolvere tre dei cinque problemi: orizzonte, piattezza e monipoli magnetici. Il problema dell'orizzonte si ha perché vediamo la connessione causale su regioni dell'universo che sono più grandi della regione dell'orizzonte e viene risolto assumendo che la connessione causale era stata raggiunta già prima. Il problema della piattezza si riferisce a una richiesta di perfetto bilanciamento tra energia cinetica e potenziale. Inoltre il modello del Big Bang prevederebbe oggi un'alta densità di monopoli magnetici che avrebbero dovuto chiudere immediatamente l'universo e comunque non sono stati osservati. I primi due problemi possono essere risolti mediante un'espansione accelerata dell'universo che inverte i comportamenti del raggio dell'orizzonte e del parametro di densità totale. La durata di questa espansione deve comunque essere sufficientemente lunga: la condizione richiesta dal primo problema $N_{ef} \gg 60$, soddisfa pienamente anche il secondo. 

Storicamente i primi modelli si sono appoggiati alle transizioni di fase perché avvengono all'energetica richiesta e generano le particelle desiderate. La dinamica è descritta dall'equazione fondamentale dell'inflazione che è derivata dall'equazione di Eulero-Lagrange per l'inflatone (parametro d'ordine delle transizioni di fase). Applicando le condizioni di slow rolling si ottengono: espansione accelerata e un numero sufficiente di e-folding. In particolare i modelli old inflation utilizzavano le transizioni del primo tipo, mentre quelli new inflation le transizioni del secondo tipo. Si è poi concluso che non funzionavano perché non riuscivano a generare regioni abbastanza grandi da contenere il nostro universo e richiedevano un fine-tuning dei parametri.

Il modello attuale è quello dell'inflazione caotica. La dinamica dell'inflatone, ora inteso come campo scalare estremamente energetico, è descritta dalla stessa equazione di cui sopra. In questo caso si può avere un ampio range di condizioni iniziali che generano quasi la stessa traiettoria, infatti inizialmente le curve cadono tutte sull'attrattore, da cui ha inizio l'inflazione (non è più richiesto un fine-tuning). Durante la fase di inflazione, $\phi_i \rightarrow \mathcal{O}(1)$, si ha espansione accelerata e si può avere l'$N_{ef}$ necessario purché $\phi_i \gg 3\div 4$. Inoltre la richiesta di non entrare in regime quantistico $V<1$ si traduce in un limite superiore a $\phi_i$. In conclusione, $ 3 \ll \phi_i < 10^4$ in unità planckiane. In questo modo si può stabilire quale valore (naturalmente al di sotto di $m_P$) associare alla massa dello scalare inflatone, e.g. $m_\phi = 10^{-6}$. Si può dimostrare che questo approccio vale anche per forme del potenziale $V$ leggermente diverse da quella quadradica. 

Quando si esce dalla fase di inflazione, $\phi = \mathcal{O}(1)$, si può innestare un meccanismo di decadimento dello scalare inflatone. La fisica delle particelle garantisce che il processo dominante è quello di decadimento in due scalari e anche pochi giri di spirale sono sufficienti per riempire l'universo di particelle. Inoltre, per avere $w=1/3$ in uscita, si può termalizzare l'universo purché la temperatura di reheating rimanga minore della temperatura GUT. Lo "stiramento" provocato dall'inflazione appiattisce l'universo e qualunque perturbazione ci sia in esso, all'uscita quindi l'universo è omogeneo e isotropo (\textit{cosmic no hair theorem}).

\vspace{1em}
\noindent Alcune predizioni dell'inflazione sono falsificabili: 
\begin{itemize}
    \item[-] $\Omega_{TOT} = 1$, universo piatto (confermato da Planck a meno di $10^{-2}$);
    \item[-] Le particelle generate creano fluttuazioni pressoché gaussiane nel campo di densità;
    \item[-] Distribuzione della scala angolare del campo di densità (pressoché \textit{scale invariant});
    \item[-] Piccole deviazioni da gaussianità e invarianza di scala legate a $\varepsilon \leftrightarrow  V' $ e $\eta \leftrightarrow V''$;
    \item[-] Fluttuazioni tensoriali (onde gravitazionali).
\end{itemize}

\vspace{1em}
In conclusione, l'inflazione non è un modello, è un paradigma, una famiglia di modelli che hanno in comune la richiesta di avere campi sufficientemente energetici. Si distinguono dall'avere predizioni diverse per $\varepsilon$, $\eta$ e onde gravitazionali. Con i dati di Plank una buona serie di modelli inflazionari è stata rigettata, ma ci sono ancora più modelli che teorici al mondo. Il vantaggio di studiare la CMB piuttosto che la LSS sta nel fatto che ha molta più memoria (perché è più temporalmente vicina) delle condizioni iniziali. 


\section{L’era Adronica e l’era Leptonica}
L'ultima delle grandi transizioni di fase, la transizione quark-adroni, avviene a $T=200\div 300$ MeV a circa $10^{-5}$ s. Da questo istante i quark si possono unire per dare origine agli adroni. 

\subsection{Era Adronica}
Le reazioni che caratterizzano questo periodo sono quelle che trasformano adroni in fotoni (reazione di annichilazione) e viceversa e l'annichilazione dei tau. Quest'era è molto breve e termina nel momento in cui anche i pioni si annichilano, $T_\pi = 130$ MeV. Alla fine si avranno: leptoni, antileptoni, fotoni e protoni e neutroni in misura pari all'eccesso barionico. In particolare, dalla distribuzione di Boltzmann:
\begin{equation}
    n=2\left(\frac{mk_b T}{2\pi \hslash}\right)^{3/2}  e^{(\mu - mc^2)/{k_B T}} \label{eq:boltzmann}
\end{equation}
dove $\mu$ è il potenziale chimico che in questo caso, come in tutti i casi in cui si hanno particelle assieme ad antiparticelle, vale $0$ (conservazione carica, numero barionico e leptonico, carica dell'universo nulla). Da cui:
\begin{equation}
    \frac{n_n}{n_p}=\left(\frac{m_n}{m_p} \right)^{3/2}e^{-(m_n-m_p)c^2 / k_B T} \label{eq:nn-vs-np}
\end{equation}
Considerando che $(m_n-m_p)c^2=1.3$ MeV $\rightarrow 1.5\cdot 10^{10}$ K, per questo periodo si può assumere $n_n=n_p$. E questa approssimazione regge fintantoché $T<1.3$ Mev.


\subsection{Era Leptonica}
Le reazioni che caratterizzano questo periodo sono quelle di produzione e di annichilazione di paia di leptoni. È un'era molto interessante perché permette di caratterizzare le proprietà dei neutrini e su di essa si fondano le condizioni iniziali della nucleosintesi primordiale. Ha inizio dal decadimento del pione ($10^{-5}$ s) e finisce quando si annichila l'elettrone ($10$ s, T=$0.5$ MeV). È caratterizzata da: leptoni ($e^-$, $e^+$, $\mu^-$, $\mu^+$, 3 neutrini), fotoni e l'eccesso barionico (trascurabile). Il peso statistico effettivo vale (eq. \ref{eq:statisticweight}):
$$
g^* = 2 + \frac{7}{8}(4\cdot 2 + 3 \cdot 2) \simeq 14.25
$$
valore molto inferiore rispetto all'era di Planck $\approx 200$ e all'inflazione $\approx 100$. Si può verificare che il tempo tipico di interazione è molto più piccolo di $\tau_{exp}$ , per cui tutte queste particelle rimangono in equilibrio termico. Successivamente si annichilano $\mu^-$ e $\mu^+$ e infine $e^-$, $e^+$.

Qualsiasi annichilazione (eg. $e^- + e^+ \leftrightharpoons \gamma + \gamma$) è un processo termodinamicamente reversibile che conserva l'entropia $S=(p+\rho c^2)V/T$, per cui considerando che tutto è accoppiato alla radiazione e il volume cambia in modo trascurabile:
\begin{equation}
    S_{-} \equiv S_{+} \qquad \rightarrow \qquad T_+ = T_- \left(\frac{g^*_-}{g^*_+}\right)^{1/3} 
\end{equation} 
e dato che $g^*_+ < g^*_-$, la temperatura dell'universo cresce in seguito a ogni fase di annichilazione. Per i fotoni, essendo dall'era di Planck ad oggi $T: 10^{32}\rightarrow 2.7$ e $g^*: 200\rightarrow 2$, l'effetto è praticamente trascurabile, ma lascia una \textit{feature} importante per i neutrini.

\subsubsection{Disaccoppiamento dei neutrini}

L'accoppiamento dei neutrini con i leptoni può avvenire attraverso le seguenti reazioni elettrodeboli:
$$
\nu_e + \mu^- \leftrightharpoons {\overbar{\nu}_\mu} + e^- \qquad {\overbar{\nu}_\mu} + \mu^+ \leftrightharpoons \nu_e + e^+ 
$$
Confrontando il tempo di interazione $\tau_{coll}=(\sigma_{EW}~n_{lep}~c)^{-1}$ con il tempo di espansione dell'universo $H^{-1}_{w=1/3} = 2t$, si ha: 
\begin{equation}
    \frac{\tau_{exp}}{\tau_{coll}}= \left (\frac{T}{3\cdot 10^{10}\;\mathrm{K}}\right)^3
\end{equation}
Per cui a partire da $T= 3\cdot 10^{10}$ K si ha il disaccoppiamento dei neutrini e avviene dopo l'annichilazione dei $\mu$, ma prima dell'annichilazione degli $e$. Questo permette di calcolare la temperatura del neutrino e quindi la massa corrispondente (assumendo che ce l'abbia). 
Ricapitolando, quando si annichila la coppia $\mu$ neutrini, leptoni e fotoni sono accoppiati e fanno ambedue un salto in temperatura. Al momento del disaccoppiamento i neutrini sono relativistici, così come i fotoni, quindi anche se non si parlano più hanno la stessa adiabatica. Al momento dell'ultima annichilazione, quella della coppia $e$ ($T=3\cdot 10^{10}$ K), soltanto la radiazione subisce un salto in temperatura dopodiché seguirà di nuovo l'andamento $a^{-1}$. 

\begin{figure}[h]
    \centering
    \includegraphics[width=.8 \textwidth]{Pictures/5/annichilazioni.png}
    \caption{Temperatura dal variare di $a$ in seguito a: decadimento del pione $a_\pi$, annichilazione dei muoni $a_\mu$, disaccoppiamento dei neutrini $a_\nu$ e annichilazione della coppia elettrone-positrone $a_e$.}
\end{figure}


La differenza di temperatura è mantenuta fino ad oggi:
$$
T_{rad} = T_- \left(\frac{11}{4} \right)^{1/3}\simeq 1.4 ~T_-
$$
La temperatura del neutrino oggi deve quindi essere:
$$
T_{0,\, \nu} \simeq \frac{T_{0,\, rad}}{1.4} \simeq 1.9 \, \mathrm{K}
$$

Se il neutrino non ha massa (modello standard), vale la relazione $\rho_\nu c^2 =\sigma T^4$, da cui si ottiene un contributo $\Omega_{0\nu}\simeq 0.7 \Omega_{0R}$ da aggiungere alla componente relativistica. 

Se il neutrino avesse massa $10$ eV (numero non a caso), si de-relativizzerebbe a $T\approx T_{eq,\, rad}$, praticamente ieri l'altro. In questo caso $T$ è una pseudo-temperatura, ma vale comunque la relazione $n_\nu \propto T_\nu^3$, che restituisce $n_\nu = 320$ cm$^{-3}$ (cfr. $n_\gamma=420$ cm$^{-3}$). Questa quantità contribuisce in questo caso alla componente di materia oscura. In particolare, assumendo che tutta la materia oscura sia dovuta ai neutrini, si può trovare un limite superiore alla loro massa media (tra i tre tipi):
$$
\left \langle m_\nu \right \rangle \le \frac{\Omega_{0m} ~\rho_{c}}{n_\nu} \qquad \rightarrow\qquad \left \langle m_\nu \right \rangle \le 10\, \mathrm{eV}
$$
In realtà sappiamo che il neutrino non ha le caratteristiche che ci piacciono per rappresentare tutta la materia oscura (è troppo "caldo") e come vedremo a breve dovrà essere minore di qualche decimo di eV. 


\section{Nucleosintesi Primordiale}
I fenomeni che caratterizzano la nucleosintesi erano già studiati negli anni '40 in associazione con gli interni stellari e le bombe atomiche. È considerata una delle prove fondamentali a favore del modello del Big Bang perché riesce a spiegare molto bene le abbondanze degli elementi leggeri (ciò che le stelline del Prof. Ferraro non possono spiegare). In particolare può giustificare l'alta abbondanza di elio $Y=0.25$. Le condizioni cosmologiche impediscono la formazione di elementi più pesanti a causa delle elevatissime temperature, il problema sarà la formazione del deuterio (che è un collo di bottiglia). Inoltre il modello predice l'esistenza di un fondo cosmico a $T=5$ K, poi scoperto 30 anni dopo. 

\vspace{1em}
\noindent Le assunzioni del modello standard della nucleosintesi primordiale sono:


\begin{table}[h]
    \def\arraystretch{1.5}
    \begin{tabular}{lll}

    \textbf{1)} & $T\ge 10^{12}$ K $\quad$  (\textit{hot Big Bang}) \\
    \textbf{2)} & General Relativity $+$ Fisica delle Particelle \\
    \textbf{3)} & Universo sufficientemente Omogeneo e Isotropo  \\
    \textbf{4)} & Cinque o meno tipologie di neutrini \\
    \textbf{5)} & Neutrini non degeneri \\
    \textbf{6)} & Non esistono troppe regioni di antimateria \\
    \textbf{7)} & $\vec{H}$ trascurabili  \\
    \textbf{8)} & Non esistono particelle esotiche \\
    \end{tabular}
    \end{table}

Le condizioni di base sono (1), (2) e (3), in particolare l'ultima è verificata con l'inflazione (paradigma non presente nei primi modelli di nucleosintesi). Le condizioni (4), (5) e (8) sono aggiunte a posteriori per evitare la sovrapproduzione di He. La condizione (6) è necessaria per evitare la sovrapproduzione di energia, mentre la (7) viene adottata per non complicare troppo i modelli.  

\subsection{Formazione del deuterio}
Questa è la fase più critica per via dell'alta probabilità dell'elio di essere fotodissociato. Come si è visto nell'equazione (\ref{eq:nn-vs-np}) vale la relazione $n_n / n_P = \exp{(-1.5\cdot 10^{10}/T)}$. L'equilibrio è mantenuto, mediante i neutrini, dalle reazioni:
$$
n + \nu_e \leftrightharpoons p + e^- \qquad n+e^+ \leftrightharpoons p + {\overbar{\nu}_e}
$$

Inoltre si ha:
$$
x_n \equiv \frac{n_n}{n_{tot}} = \frac{n_n}{n_n + n_p} \simeq 0.17
$$
I neutroni rappresentano il 17\% del totale delle particelle finché l'equilibrio tra neutroni e protoni è mantenuto. In realtà anche dopo il disaccoppiamento dei neutrini ($T_{D\nu}\approx 10^{10}$ K) vi sono reazioni residue che garantiscono $x_n (0)=0.17$ fino a $t_N=20$ s ($T_N=1.3 \cdot 10^{9}$ K). Dopo questo tempo dominerà il processo di decadimento del neutrone $ n \rightarrow p + e^- + {\overbar{\nu}_e}$ con un $\tau_n \approx 900$ s, ossia si avrà:


\begin{equation*}
    x_n = x_n (0)~ e^{(t-t_N) / \tau_n}\simeq 0.17 ~e^{(t-20)/ 900}
\end{equation*}

\vspace{1em}
\noindent Per la produzione del deuterio deve avvenire la reazione: $n+p \leftrightharpoons D + \gamma$ molto sfavorita dalla fotodissociazione. Si parte dall'equaizone di Boltzmann (\ref{eq:boltzmann}), ma in questa fase $\mu\neq 0$ perché non ci sono più antiparticelle. Rispettando le regole di ingaggio che seguono e considerando che $g_D=3$ e $(m_n+m_p-m_D)c^2=2.2$ MeV, si ottiene:
\begin{equation*}
    \mu_n + \mu_p = \mu_D +  (\mu_\gamma =0) \quad \rightarrow \quad x_D=x_n ~x_p \exp{\left(-29.33+\frac{25.82}{T_9} -\frac{3}{2}\ln T_9 - \ln (\Omega_b h^2) \right)} 
\end{equation*}
che relaziona la quantità di deuterio in funzione della quantità di $n$ e $p$; in particolare:

\begin{equation}x_D \approx \left\{
    \def\arraystretch{1.5}
        \begin{array}{ll}
            0 & T_9 \gg 1 \\ 
            x_n ~x_p & T_9 = 0.9 \quad\mathrm{per}\quad \Omega_b \simeq 10^{-2}
    \end{array}\right. \label{eq:xdeuterio}
\end{equation}

Il deuterio inizierà ad essere significativo quando la temperatura $T_9=0.9$, ossia è tanto alta da contrastare il fattore numerico $-29.33$ e questo avviene a $t^*=200$. Ne caso in cui $\Omega_b \simeq 1$ si avrebbe $T_9=0.9$ a $t^*=300$ s.

\subsection{Formazione dell'elio}
A partire da circa 200 secondi (4 minuti) dopo il Big Bang c'è sufficiente deuterio affinchè - istantaneamente - abbiano inizio le reazioni di produzione dell'elio:
$$
D + D \leftrightharpoons {^3He} + n \qquad {^3He} + D \leftrightharpoons {^4He} + p 
$$
L'abbondanza in massa dell'elio vale:
\begin{equation}
    Y = \frac{m_{He}}{m_{tot}} = 4 \frac{1}{2} \frac{n_n}{n_{tot}} = 2 x_n(t^*)
\end{equation}
Utilizzando le relazioni e $t^*$ trovati in precedenza:
$$
Y\simeq 0.25
$$
Valore che soltanto in parte ($\sim 1/6$) può essere prodotto dagli interni stellari; questo è stato il successo del modello della nucleosintesi. Inoltre si può notare che questo valore dipende poco da $n_{tot}$ perché dominano le interazioni elettrodeboli e non quelle tra nucleoni, motivo per cui funziona con diversi modelli cosmologici ($\Omega_b$). È la temperatura che trigghera la formazione dell'elio. 

Le densità dell'univiverso erano tali da non permettere ulteriori reazioni (${^8Be}$, {$^{12}C$), si forma soltanto un po' di $^7{Li}$. 

In conclusione, l'${^4He}$ è praticamente costante al variare di $\Omega_b$ con un piccolo trend di crescita dovuto al fatto che più si hanno barioni, prima parte la produzione del deuterio e la produzione di ${^4He}$ è più efficiente. L'abbondanza del deuterio, al contrario, è estrememente sensibile: varia di $8$ dex, variando $\Omega_b$ di $3$ dex.

